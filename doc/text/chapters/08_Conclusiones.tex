\chapter{Conclusiones y trabajos futuros}

\section{Conclusiones}

El objetivo de este {\sf Trabajo Fin de Máster} era el de abordar el problema de la obtención de información de los datos contenidos en un portal de datos abiertos mediante peticiones a una interfaz en forma de consulta, de forma que pudiéramos sacar conclusiones sobre la información que ahí estaba contenida.
\bigskip

Para poder extraer el conocimiento que hubiera contenido necesitaríamos que el propio sistema nos ayudara procesando los datos de forma automáticamente de forma que nos fuera más útil entenderlos, pero para esto hay que tener en cuenta un aspecto muy importante; las máquinas no conocen el significado de las cosas, solo sus valores, por lo que tenemos que hacer que los datos sean entendibles por las máquinas.
\bigskip

Como en esta línea de acción se lleva trabajando desde hace bastante tiempo, hemos podido basar todo el desarrollo en estándares que aunque no tengan una gran penetración en el día a día de la {\sf Web}, si que tienen capacidad para hacer cosas grandes. Y esto es así porque el objetivo que se persigue es mucho más ambicioso: hacer que todos los portales de datos de Internet estén interconectados y sus datos vinculados de forma se favorezca la interoperabilidad entre plataformas distintas, pero que seguirán unas mismas estructuras definidas según el campo de actuación.

\newpage
Además, esto también dotaría a las personas de adquirir un mayor conocimiento de una forma más sencilla para ellos al contar con la ayuda de las máquinas. Igualmente, la tendencia es que cada vez más instituciones abran todos sus datos, por los que este tipo de estándares facilitan el acceso al mismo, ya que se puede acceder fácilmente mediante interfaces web que funcionan como capa de abstracción sobre los sistemas gestores de bases de datos que pueden resultarle más difíciles de entender a personas que no tengan un conocimiento informático previo.

\bigskip
El caso particular de este proyecto, se han presentado varias dificultades ya que aunque la ontología diseñada ha permitido que el sistema funcione perfectamente; ha sido la ontología la que se ha tenido que adaptar a una estructura de datos ya existente de forma anterior. Sin lugar a dudas, lo ideal hubiera sido poder partir de un sistema totalmente vacío e ir decidiendo como modelar los datos y qué se busca exactamente compartir.

\section{Trabajos futuros}

Como comentábamos en las conclusiones, una posible extensión de este trabajo sería desarrollar un sistema desde la nada, para adaptarlo a las necesidades que se consideren oportunas después del estudio pertinente.

\bigskip
En cualquier caso, no todas las posibilidades de ampliación llevan por hacer el trabajo desarrollado en este {\sf Trabajo Fin de Máster} quedaran en nada, hacer un sistema más personalizado y cambiar la gestión de los {\sf URI} para cada uno lleve a páginas individuales de los recursos sería también una gran ampliación, además eso haría que el sistema tuviera un gran parecido a un referente en este aspecto como es la DBpedia.