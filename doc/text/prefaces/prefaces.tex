\begin{center}
{\LARGE\bfseries\titulo}\\
\end{center}
\begin{center}
\autor\
\end{center}

\textbf{Palabras clave: }{\keywordsEs}

\section*{Resumen}

En este Trabajo de Fin de Máster se ha propuesto transformar conjuntos de datos que se consideren relevantes de los disponibles en el portal de datos abiertos de de la Universidad de Granada, OpenData UGR, a un formato que permita que la información contenida en dichos datos sea comprensible y razonablemente automáticamente por máquinas. El objetivo final de esto es tener una interfaz que, a través de la ejecución de consultas, devuelva esta información ya procesada por la máquina.

\bigskip
Para resolver este problema se diseñara una ontología para describir y representar la información contenida en los datos almacenados. Esto se hará utilizando diferentes estándares definidos por el World Wide Web Council (W3C) para la definición de la web semántica como son RDF, RDFS y OWL. Será necesario transformar los datos originales (con diferentes distribuciones, todos en formato CSV) en un formato compatible con este sistema, principalmente RDF / XML, aunque también se utilizará el formato Turtle (Terse RDF Triple Language). Para ello, se usarán scripts escritos en Python para ese propósito.

\bigskip
Una vez que todos los datos son procesados, sólo será necesario proporcionar un punto final a un sistema de recuperación de datos utilizando SPARQL, que en este caso será montado en OpenLink Virtuoso, un servidor abierto ORDBMS que permite el almacenamiento y gestión de datos en formato RDF.

\newpage
\begin{center}
{\LARGE\bfseries\tituloEng}\\
\end{center}
\begin{center}
\autor\
\end{center}

\textbf{Keywords: }{\keywordsEn}

\section*{Abstract}

In this Master's End Work it has been proposed to transform data sets that are considered relevant from those available in the open data portal of the University of Granada, OpenData UGR, to a format that allows the information contained in such data to be understandable and reasonable automatically by machines. The final objective of this is to have an interface that, through the execution of queries, returns this information already processed by the machine.
\bigskip

To solve this problem an ontology will be designed to describe and represent the information contained in the stored data. This will be done using different standards defined by the World Wide Web Council (W3C) for the definition of the semantic web as they are RDF, RDFS and OWL. It will be necessary to transform the original data (with different distributions, all in CSV format) into a format compatible with this system, mainly RDF / XML, although the Turtle (Terse RDF Triple Language) format will also be used. For this, scripts written in Python will be used for that purpose.

\bigskip
Once all the data is processed, it will only be necessary to provide an endpoint to a data recovery system using SPARQL, which in this case will be mounted on OpenLink Virtuoso, an open ORDBMS server that allows the storage and management of data in RDF format.

\newpage
\thispagestyle{empty}
\
\vspace{3cm}

\noindent\rule[-1ex]{\textwidth}{2pt}\\[4.5ex]

Yo, \textbf{\autor}, alumno de la titulación \textbf{\master} de la \textbf{\escuela\ de la \universidad}, autorizo la ubicación de la siguiente copia de mi Trabajo Fin de Máster (\textit{\titulo}) en la biblioteca del centro para que pueda ser consultada por las personas que lo deseen.

\bigskip
Además, este mismo trabajo es realizado bajo licencia \textbf{Creative Commons Attribution-ShareAlike 4.0} (\url{https://creativecommons.org/licenses/by-sa/4.0/}), dando permiso para copiarlo y redistribuirlo en cualquier medio o formato, también de adaptarlo de la forma que se quiera, pero todo esto siempre y cuando se reconozca la autoría y se distribuya con la misma licencia que el trabajo original. El documento en formato {\tt LaTeX} se puede encontrar en el siguiente repositorio de {\tt GitHub}: \url{https://github.com/germaaan/ProyectoDatosAbiertos}.

\vspace{4cm}

\noindent Fdo: \autor

\vspace{2cm}

\begin{flushright}
\ciudad, a \today
\end{flushright}

\newpage
\thispagestyle{empty}
\
\vspace{3cm}

\noindent\rule[-1ex]{\textwidth}{2pt}\\[4.5ex]

D. \textbf{\tutor}, profesor del \textbf{Departamento de Arquitectura y Tecnología de Computadores} de la \textbf{\universidad}.

\vspace{0.5cm}

\vspace{0.5cm}

\textbf{Informa:}

\vspace{0.5cm}

Que el presente trabajo, titulado \textit{\textbf{\titulo}}, ha sido realizado bajo su supervisión por \textbf{\autor}, y 
autoriza la defensa de dicho trabajo ante el tribunal que corresponda.

\vspace{0.5cm}

Y para que conste, expide y firma el presente informe en \ciudad\ a \today.

\vspace{1cm}

\textbf{El tutor:}

\vspace{5cm}

\noindent \textbf{\tutor}

\chapter*{Agradecimientos}
\thispagestyle{empty}

\vspace{1cm}

A toda la gente de las comunidades que me rodean y que me hace descubrir todos los días cosas nuevas.